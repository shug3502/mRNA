\documentclass[a4paper,10pt]{article}
\usepackage{natbib}
\usepackage{amsmath}
\usepackage[utf8]{inputenc}

%opening
\title{Parameters from direct measurement}
\author{Jonathan Harrison}

\begin{document}

\maketitle

\section{Model parameters}
All data obtained by Alex Davidson from measurements of grk-GFP mRNA complexes in \textit{Drosophila} nurse cells.
Particles are classed as static, paused or active.
The number of total particles counted was $N=340$ in a 40x40 $\mu \text{m}$ area.
The proportions of particles in each of the classes is $25\%$ active, $50\%$ paused and $21\%$ static. 
Over the timescale observed, particles did not transition between states and these are proportions of particles. 

Particle speeds were assessed by observing a 10x10 $\mu \text{m}$ area for 50 timepoints, with images taken in 1 $z$ slice at 3 frames per second. Average speed in active transport $\nu_1 $ was  $1.163 \pm 0.08 \mu \text{ms}^{-1}$ from $n=33$. 
Average run length was $2.785 \pm 0.66 \mu \text{m}$. 

In the paused phase, an average speed,$\nu_2$, of $0.798 \pm xxx \mu \text{ms}^{-1}$ was observed from 58 particles with an average run length of  $0.84 \pm 0.06$. 

Combined particle movement (excluding static phase) gives an average run speed of $0.931 \pm 0.05 \mu \text{ms}^{-1} $  and an average run length of $1.546 \pm 0.3 \mu \text{m} $ from $n=91$ particles.
The maximum particle speed recorded was $2.795 \mu \text{ms}^{-1}$ and  the longest run length was $21.33 \mu \text{m}$. 

This gives parameter values to use of $N=340$, $\nu_1 = 1.16$, $\nu_2 = 0.80$, $\omega_1 = 0.42 $, $\omega_2 = 0.84$, $\lambda_1 = 0$, $\lambda_2 = 0.11$ .
If we are only interested in the simplest model, then we only look at the fastest active transport (looking at the difference in speeds, maybe not a great approximation). 
This gives $\lambda = \frac{1.163}{2.785} = 0.42$
Otherwise we need to distinguish between transitions between states and internal transitions. 

\section{Markov chain steady states}

Considering the process as a continuous time Markov chain, where we have split the diffusion state into two sub-states, we can obtain the steady states.

Continuous time markov chain steady state is $[\frac{\omega_2 }{\omega_1 + \omega_2},\frac{ \omega_1}{2(\omega_1+\omega_2)}, \frac{\omega_1}{2(\omega_1+\omega_2)}]$. 

Then assuming ergodicity and neglecting anchored particles (is this a valid assumption?), data gives us $\frac{\omega_2 }{\omega_1 + \omega_2} = \frac{2}{3}$ and $\frac{\omega_1 }{\omega_1 + \omega_2} = \frac{1}{3}$.
So the rates of falling off and reattaching must be different and $\omega_2 = 2 \omega_1$.

Now assuming that $\lambda_1$ is 0, then we have $\omega_1 = 0.42$ and hence $\omega_2 = 0.84$. 
Then the rate $\lambda_2 = 0.95-0.84 = 0.11$. 
 

\bibliographystyle{plainnat}
\bibliography{my_citations.bib}


\end{document}